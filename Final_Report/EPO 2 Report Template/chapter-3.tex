\chapter{ General Requirements for the EE1L21 Report}


\subsection{ Design Report}

  The final report is close to a \textit{Design type of Reporting (i.e., a design report)}. This is one of many types of reports; other types include \textit{literature report}, \textit{laboratory report}, \textit{progress report }and \textit{advisory report}. Each of these report types has certain standard characteristics and pieces of content that are needed to fulfil expectations of the report concerning \textit{purpose and scope }as indicated above under General Requirements. The purpose of a Design Report is to document and communicate a specific design problem and its solution. Its scope is just that specific problem - and not some other problem. Among the readers (the one who should understand the material that you write) in real life can be the project client, but also your management, your fellow team members, you yourself (after some time has passed and you need to catch-up again), a subsequent development team that continues with a follow-up product, and so on.

  A design report is not a chronological account of the process that you went through to arrive at the final solution - it is not a beautification of your journal / log book. Instead, it is a structured description from initial problem to final solution, including an evaluation of the quality of the solution and a discussion. The following is a typical breakdown of sections/chapters in a design report, see e.g. \cite{Elling}.

\begin{enumerate}
\item  \textit{The introduction}, a somewhat high-level, coarse and global description of the 'thing to be designed', what function needs to be fulfilled, with a description of the overall characteristics.

\item  \textit{The requirements}, what are the quantifiable targets to be reached by the described module.

\item  \textit{The followed design methodology}, describing the possible design options and motivating the choices adopted.

\item  The \textit{implementation for the oncampus (and for the online activities, mainly the simulations, use of the developed robot simulator, and functioning of your c and VHDL codes)}, detailing the specific of the developed module. And highlighting when required the differences from literature or course material implementations.

\item  The \textit{testing phase for the oncampus tasks (and for the online activities, mainly the simulations, use of the developed robot simulator, and functioning of your c and VHDL codes)}, reporting quantitative results, providing a list of the used equipment used and explaining how the testing was carried out and reporting the results. Include illustrating snapshots of waveforms (please pay attention to the legibility of the plots and/or images and ensure the proper contrast of the figures in both electronic and printed format).

\item  A \textit{critical discussion }on the obtained results. Please note that the schedule above must be mapped intelligently on a structure with chapters and sections, depending on the amount of information to be presented. In short, it can be perfect to include the first three items in an Introduction chapter or section. In more extensive works, this could be split in either two or three parts. Also note that the introduction typically also contains other standard elements of a report, such as the outline of the report with explicit reference to the different chapters and/or sections. It may also need to include a description of the key concepts required by the reader to understand the report.

\end{enumerate}



\subsection{ Citation}

\begin{enumerate}


\item  General Principles of Citation

  The BB module `Information Literacy 1' has introduced you to the basic principles of finding and citing other work, by way of preparation for the citing you have to do in your EE1L21 report. You should make use of what you learned in IV1 in your EE1L21 final report. Please find motivation and further elaboration of the role and manners of citing in academic writing, including your EE1L21 final report, below. While designing, you will make use of artefacts and technologies that already exist, and combine those to accomplish your design challenge. While designing, you will need to refer to the documentation of those artefacts and technologies to know how to use them. For fulfilling the main premise of your design report, namely documenting your design, you need to specify those background documents in the design report. This is done through \textit{citing }those documents. Side note: citing is also needed in other types of documents, such as scientific papers and other research documents. In that case, citing serves the traceability of ideas - it must always be clear what is the origin of any concept or idea that is described in a research paper. You will learn more about citing in other types of documents in subsequent reporting assignments in the BSc EE program. In general, please accept that through correct citation, you show how you build upon the work of others. Building upon the work of others is not a form of weakness, on the contrary. Science and technology can only make progress by using the work of those who came before you as a starting point. This calls for complete traceability of ideas, which in itself is an important reason for citing correctly. Another important reason is one of ownership of ideas: people should receive credit when others use their \textit{intellectual property}. Furthermore, please realize that your work becomes more credible and stronger if you use strong, high-quality and authoritative references. The need for citation extends to the use of figures, tables or other fragments of source material. Not doing so is a form of plagiarism, which is heavily charged in science and can lead to disqualification. Play it safe. Cite all material that you include in your report that comes from other sources. If it is a direct inclusion of short pieces of text from another document, put it between quotes and add a citation. If this is a picture/graph or similar, you can add a citation in the caption of the figure (and in the text where you first refer to that figure). You must also cite if you use an original idea or concept from another source, even if you describe it in your own words (paraphrasing).


\item  Citation using IEEE style

  While the IV1 module has mentioned other citation styles, the preferred style of citation in Electrical Engineering typically is the so-called IEEE style. The IEEE is the largest international professional association in the field of Electrical Engineering. As such, it is also a publisher of many scientific journals and other periodicals, conference proceedings and standards, and very influential at this. The so-called \textit{IEEE Style }is in many cases the preferred style guide for many types of documents in the field. Formatting of citations and bibliographies is an important element of style, and as such the IEEE citation style \cite{IEEE}) is preferred for the EE1L21 reporting (and for many other reports and documents during your education, usually inclusive of your BSc thesis, for that matter). You are not required to follow the exact detailed rules as outlined in the official IEEE style guides (such as \cite{IEEE}, also cited above), but it \textit{is} recommended to follow the most distinctive aspects of this style:


\item  Citations are numbered, and included in the text between square brackets (not superscripts).

\item  The bibliography is a list of references at the end of the document, ordered in order of


  appearance in the text, and numbered with the same square brackets.

\item  Author names are included, as the first element(s) of the bibliography item, usually \textit{initials, last name}. See for example reference \cite{Elling}. See the citation style and references in this manual, and generally in all works published by the IEEE, as an example. The information that has to be included in the bibliography items must make the citation precise and should aid retrieval/searching for the document. In general, the information to be included is the same as in other styles such as APA.

\end{enumerate}

  


\subsection{ Simulation/Measurement Errors}

  

  In the EE1L21 final report, you have to pay attention to the correct reporting of simulation/measurement results, in view of the accuracy of the simulation/measurement results. It makes no sense to report a quantity with 5 digits behind the decimal point if the error could be plus or minus 5\%.

  




