\chapter{Risks Analysis }
In this chapter, we will review several risks with a high likelihood of occurring and how to counter or resolve the possible issues that might ensue.\\
\\
The risks are divided in three categories based on their main cause.
\section{Causes}
\subsection{Time related risks}
Time-related risks involve the possibility of a team not fulfilling their work in time, which results in missing important deadlines. This can be caused by: miscommunication among team members, procrastination, misunderstanding of the assignment, lack of motivation.

\subsection{Member related risks}
These risks are caused by a team member not fulfilling their workload or contributing to the project, which can be caused by a lack of time, motivation or laziness. Unexpected personal issues are part of member related risks. Even small occurrences as missing meetings and not attending lab sessions can fall under member related risks.

\subsection{Communication related risks}
These risks are caused by bad communication. For example a miscommunication of who will finish the report or working on a problem that is already finished or unnecessary. The unclear communication between members can lead to small disputes. In the worst case, those might escalate into quarrels that leave long-lasting grudges between members. 

\section{Avoiding and solving risks and issues}
\subsection{Time related risks}
An easy way to avoid time-related risks is to keep track of the progress within the group and make good planning at the start of the project. 

In this planning, we clearly state the expectations and deadlines so everyone can clearly see what has to be done and when. This way, we can avoid anyone misreading or misunderstanding an assignment.
Expectations and deadlines will made clear and there will be consequences, which are elaborated at member-related risks
\\\\

It is also important to divide the tasks so everyone knows what they have to do. The project will be divided into smaller parts and a small subgroup of people will be assigned to work on those tasks. The division of tasks will be done at the start of the project where everyone can pick and choose the areas they want to work on. We only finalize the planning with unanimous approval from everyone so nobody will feel like their work is unfair.\\\\

The planning and task-division is in order to make sure that the project will be finished on time. However, checks and balances are also needed in between working on the project.\\\\

At the start of every meeting, the chair has the responsibility to look up all the upcoming deadlines and remind everyone of them. He will also have to know what everyone's task is, so he can clearly remind everybody of their duties that day. During the session, the chair  has the duty to assign new tasks to subgroups that are ahead of schedule (this will usually involve helping other subgroups). At the end of each session, every subgroup has to give a short presentation on what they have accomplished in that session and what still has to be done. If a group has fallen behind, the secretary will write that down, and the chair will discuss a solution with that subgroup. This keeps every subgroup accountable for the planning. \\\\
When the whole group is clearly behind schedule, it is the duty of the chair to plan an emergency meeting and propose a solution to the group. This should be enough to keep everybody on track.\\

\subsection{Member related risks}
These issues can be countered by following up on the progress of each individual and assessing their input. If it is noticed that a member doesn't take his responsibilities seriously, actions will be taken.\\\\
Members will be held accountable if they are frequently late or miss meetings. Attendance will  be noted for important meetings, as well as being late without a good reason. The attendance will be checked by the chair. He will relay this information to the secretary who will note this down in a public list. \\\\ A member could also compromise the group by delivering work that is unfinished or sub-par. Therefore it is important that everyone's work gets checked by the rest before submission. Before handing anything in, everyone has to go over everyone's work. Nothing will be submitted without unanimous approval. \\\\
Each subgroup will choose its own specific way of penalizing members. Penalties will be handed out on a case-by-case basis. This is important as many circumstantial factors might be at play (someone is late because of the NS, or traffic). It is important to remember that sub-member work is not necessarily the result of ill intent. Possible penalties might include: having them finish the work on the weekend, assigning them a lower grade at the end, and expulsion from the project [Worst case].
\\
In the case that a member would drop out, the workload will have to be redivided among the remaining members to the best of our ability.
\subsection{Communication related risks}
These risks can be prevented by making a good plan, and writing it down. Everyone should check if they clearly understand what their tasks and duties are before finalizing any plan\\\\ Communication will mainly happen with each other on WhatsApp and during lab sessions on Monday and Thursday. Here it is the responsibility of the chair to lead the conversation. The chair has to keep everyone on topic and make sure that everyone's voice is heard. At the end of every meeting, everyone has the right to quickly address something that the chair might have forgotten or overlooked.\\\\ What is also important is that if there are people who feel they might drop out. They should inform the other group members about this, so the other group members can anticipate this. If someone from a specific subgroup drops out, someone from another subgroup can fill in for that role. If there are too many dropouts, the TA and mentor will be contacted about this, otherwise, it would be impossible to finish the project on time and at all.\\\\ Finally, if any disputes develop between members that are on the verge of escalating, the chair has the responsibility to talk with them in private to resolve them as soon as possible. In the case of no agreement, the members could be separated into different subgroups. Anything worse would be referred to counseling by the university. Once again we highly doubt that any of this will ever be necessary.\\

\section{Conclusion of risks analysis}
Most, if not all of the expected and probable risks are easy to avoid with clear planning and a clear task allocation that everyone follows. It is important to have a clear power structure that keeps the individual subgroups accountable to that planning and ensures the quality of work.  