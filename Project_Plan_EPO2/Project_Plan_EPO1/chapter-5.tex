\chapter{Risks Analysis }
In this chapter, we will review several risks with a high likelihood of occurring and how to counter or resolve the possible issues that might ensue.\\
\\
The risks are divided in three categories based on their main cause.
\section{Causes}
\subsection{Time related risks}
This is mainly the risk of missing deadlines due to unfinished work. This could be caused by: miscommunication between members on who does what, misunderstanding the assignment, procrastination of work, and individual members' laziness, among other things.
\subsection{Member related risks}
These are risks related to a member failing to take up his workload and part of the project and can be caused by lack of time, motivation, or plain laziness.\\
There is also the possibility of unexpected personal issues.\\
Smaller things happening like missing meetings or not attending lab sessions can also fall under member-related risks.
\subsection{Communication related risks}
A lot of mistakes could be made, due simply to bad communication.\\
One group might be designing a component based on different output or input specifications than another group is expecting thus resulting in incompatible systems. Someone might expect another group member to finish the report while the other thinks the opposite. Someone could be working on a problem that is already finished or unnecessary. In the worst case, small disputes might escalate into quarrels that leave long-lasting grudges between members. This would obviously be terrible for the project.\\
The common denominator between these problems is unclear communication between members.
\section{Avoiding and solving risks and issues}
\subsection{Time related risks}
An easy way to avoid time-related risks is to keep track of the progress within the group and make good planning at the start of the project. In this planning, we clearly state the expectations and deadlines so everyone can clearly see what has to be done and when. This way, we can avoid anyone misreading or misunderstanding an assignment. \\\\
It is also important to divide the tasks so everyone knows what they have to do. We divide the project into smaller parts and assign a small subgroup of people to work on those tasks. The division of tasks will be done at the start of the project where everyone can pick and choose the areas they want to work on. We only finalize the planning with unanimous approval from everyone so nobody will feel like their work is unfair.\\\\
All of this should be enough to make sure that the project will be finished on time, but we need more checks and balances to be sure.\\\\
At the start of every meeting, the chair has the responsibility to look up all the upcoming deadlines and remind everyone of them. He will also have to know what everyone's task is, so he can clearly remind everybody of their duties that day. During the session, the chair  has the duty to assign new tasks to subgroups that are ahead of schedule (this will usually involve helping other subgroups). At the end of each session, every subgroup has to give a short presentation on what they have accomplished in that session and what still has to be done. If a group has fallen behind, the secretary will write that down, and the chair will discuss a solution with that subgroup. This keeps every subgroup accountable for the planning. \\\\
When the whole group is clearly behind schedule, it is the duty of the chair to plan an emergency meeting and propose a solution to the group. This should be enough to keep everybody on track.\\

\subsection{Member related risks}
These issues can be countered by following up on the progress of each individual and assessing their input. If it is noticed that a member doesn't take his responsibilities seriously, actions will be taken.\\\\
Members will be held accountable if they are frequently late or miss meetings. Therefore attendance will be noted for important meetings, as well as being late without a good reason. Checking attendance is the duty of the chair. He will relay this information to the secretary who will note this down in a public list. \\\\ A member could also compromise the group by delivering work that is unfinished or sub-par. Therefore it is important that everyone's work gets checked by the rest before submission. This is easy for the filter groups as their work is quite similar so they can easily understand and check each other's work. For the other subgroups it is important that they communicate that their work is explained clearly at the end of each session (see: 5.2.1). Before handing anything in, everyone has to go over everyone's work. Nothing will be submitted without unanimous approval. \\\\
Each subgroup will choose its own specific way of penalizing members. Penalties will be handed out on a case-by-case basis. This is important as many circumstantial factors might be at play (someone is late because of the NS, or traffic). It is important to remember that sub-par work is not necessarily the result of ill intent. Possible penalties might include: having them finish the work on the weekend, assigning them a lower grade at the end, and expulsion from the project [Worst case].
\\
In the case that a member would drop out, the workload will have to be redivided among the remaining members to the best of our ability.
\subsection{Communication related risks}
Once again this is an issue that is easy to prevent by making a good plan, and writing it down. Everyone should check if they clearly understand what their tasks and duties are before finalizing any plan\\\\ Communication will happen with each other on whats app, discord, and in the meetings on Tuesday. Here it is the responsibility of the chair to lead the conversation. The chair has to keep everyone on topic and make sure that everyone's voice is heard. At the end of every meeting, everyone has the right to quickly address something that the chair might have forgotten or overlooked.\\\\ What's also important is that if there are people who have a feeling they might drop out. They should inform the other group members about this, so the other group members can anticipate this. So if someone from a specific subgroup drops out we can put someone from another subgroup on it. If in the end there are too many dropouts we should contact the TA and mentor about this, otherwise, it would be impossible to finish the project on time and at all.\\\\ Finally, if any disputes develop between members that are on the verge of escalating, the chair has the responsibility to talk with them in private to resolve them as soon as possible. This is hopefully never needed, but in the case of no agreement, the chair might decide to separate the members into different subgroups. Anything worse would be referred to counseling by the university. Once again we highly doubt that any of this will ever be necessary.\\


\section{Conclusion of risks analysis}
Most, if not all of the expected and probable risks are easy to avoid with clear planning and a clear task allocation that everyone follows. It is important to have a clear power structure that keeps the individual subgroups accountable to that planning and ensures the quality of work.  