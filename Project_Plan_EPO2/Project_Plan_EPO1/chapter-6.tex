\chapter{Conclusion} 
Our goal is to construct a loudspeaker system with a flat response. The project group is divided into a few subgroups that each work on different parts of the system. \\
\\
We are trying to achieve this goal so we can get a better understanding of the theoretical concepts that were given to us in the first semester. We also want to learn about how to work together professionally and get better at certain academic skills. \\
\\
To ensure success. We divided the project group into several subgroups in order to achieve the previously mentioned goal more efficiently. We also assigned a chair and a secretary for each week. This is so we can all learn what it's like to be chair of a meeting. We also made a plan for everything that should be done each week. \\
\\
Our project also has limits. For example, the quality of the speaker is limited by the quality of the components that are given to us, and our ability to complete this project is directly dependent on our ability to work as a team. We also are not experienced in building speakers yet. \\
\\
At last, we perform a risk analysis on everything that could be harmful to our final product and to our ability to work together. These include time-related risks, risks relating to members of the project group, and communication-related risks. Most of these risks are however quite easy to avoid and we are sure we should be able to do that. \\
\\
This project might have seemed daunting at first, but after dividing the project up into more manageable goals, we are confident of our success. However, we must be careful not to fall into the same traps that countless other projects fall into and fail. \\
\\
We must stay strong and work diligently as a team to create the best speaker possible. But these promises now are just words. We must follow up on them  and make the results speak for themselves.\\