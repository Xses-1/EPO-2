\chapter{Project activities}
In order to achieve the project's results, the group must divide the tasks well and communicate with each other clearly. In other words, work distribution has to be fair and meetings must be held regularly. A detailed description of the workload distribution with a timeline will be found in the next chapter.\\
\\
Regarding communication and file management, the group will use a shared Google Drive and WhatsApp. Each subgroup will upload each week's measurements, files, and calculations in the Drive.
In this chapter, we will briefly talk about what each group member will be doing and the roles of the meetings of the team. To finish the chapter, we will talk about the project limits.

\section{Task distribution}
As mentioned, efficient task distribution is key in order for the project to end well. Below you will find the subgroups of our project. Furthermore, in the appendix, you can find a compact work breakdown structure (figure 7.1) of each of the subgroups.
\subsection{Task distribution of A1\textunderscore1} 
\begin{table}[h]
\centering
\setlength{\arrayrulewidth}{0.5mm}
\setlength{\tabcolsep}{19pt}
\renewcommand{\arraystretch}{1.5}
    \begin{tabular}{|c|c|}
    \hline
    Power Supply     &  Maximiliaan Sobry, Ruben Noort\\
    \hline
    Power Amplifier     &  Thijs van Esch, Kevin Pang   \\
    \hline
    Low pass filter & Jeroen van der Werf, Dani van den Akker\\
    \hline
    Mid filter & Thijs Prakken, Wilson Rong \\
    \hline
    High pass filter & Pieter Olyslaegers \\
    \hline
    \end{tabular}
    \captionsetup{justification=centering}
    \caption{Subgroups A1\textunderscore1}
    \label{tab:my_label}
\end{table}
\newpage
\clearpage
\subsection{Task distribution of A1\textunderscore2}

\begin{table}[h]
    \centering
    \setlength{\arrayrulewidth}{0.5mm}
    \setlength{\tabcolsep}{19pt}
    \renewcommand{\arraystretch}{1.5}
    \begin{tabular}{|c|c|}
    \hline
    Power Supply     &  Adam Amnouh, Eliskan Karayi\v git\\
    \hline
    Power Amplifier     & \hspace{50pt}Wiktor Tomanek \hspace{50pt}   \\
    \hline
    Low pass filter & Darren van de Pol, Eojin Lim\\
    \hline
    Mid filter & Andy Zhang, Ishaan Sadal \\
    \hline
    High pass filter & Gijori Atmopawiro, Steven Bos \\
    \hline
    \end{tabular}
    \captionsetup{justification=centering}
    \caption{Subgroups A1\textunderscore2}
    \label{tab:my_label}
\end{table}


As mentioned, in the next chapter there will be a more detailed planning of each subgroup per week.


\section{Roles during meetings}
To have organized and useful meetings there must be roles during each meeting. Each group member will be the secretary and chairman for a week. Week 2.1 is not in the table, because that week was an introduction to this project, so no meetings were held.\\
\\
The chairman will lead and initiate meetings. On top of that, he will make sure that each subgroup is on track and that everyone is doing his job accordingly. The chairman is also the person to solve conflicts if they occur.\\
\\
The secretary will keep track of all the files and be responsible for the minutes during each meeting. Also, he will note all the progress, meetings and activities conducted by the group during that week. This way there is a clear overview of what has been done during the quarter and the progress made.
\newline

\begin{table}[!h]
\centering
\setlength{\arrayrulewidth}{0.5mm}
\setlength{\tabcolsep}{19pt}
\renewcommand{\arraystretch}{1.5}
    \begin{tabular}{|c|c|c|}
         \hline
         \makecell[l]{Week} & Chair & Secretary\\
         \hline
         2.2 & Dani & Wilson\\
         \hline
         2.3 & Wilson & Pieter\\
         \hline
         2.4 & Pieter & Jeroen\\
         \hline
         2.5 & Jeroen & Kevin\\
         \hline
         2.6 & Kevin & Maximiliaan\\
         \hline
         2.7 & Maximiliaan & Ruben\\
         \hline
         2.8 & Ruben & Thijs P.\\
         \hline
         2.9 & Thijs P. & Thijs v. E.\\
         \hline

    \end{tabular}
    \captionsetup{justification=centering}
    \caption{Roles during meetings A1\textunderscore1}
    \label{tab:Roles during meetings A1\textunderscore1}
\end{table}

\begin{table}[h]
\centering
\setlength{\arrayrulewidth}{0.5mm}
\setlength{\tabcolsep}{19pt}
\renewcommand{\arraystretch}{1.5}
    \begin{tabular}{|c|c|c|}
         \hline
         \makecell[l]{Week} & Chair & Secretary\\
         \hline
         2.2 & \hspace{10pt} Wiktor \hspace{10pt} & Mati\\
         \hline
         2.3 & Steven & \hspace{12pt} Gijori \hspace{12pt} \\
         \hline
         2.4 & Gijori & Ishaan\\
         \hline
         2.5 &  Ishaan & Andy\\
         \hline
         2.6 &  Andy & Darren\\
         \hline
         2.7 &  Darren & Eojin\\
         \hline
         2.8 &  Eojin & Eliskan\\
         \hline
         2.9.1 &  Eliskan & Adam\\
         \hline
         2.9.2 & Adam & Wiktor\\
         \hline

    \end{tabular}
    \captionsetup{justification=centering}
    \caption{Roles during meetings A1\textunderscore2}
    \label{tab:Roles during meetings A1\textunderscore2}
\end{table}

\newpage
\section{Project limits}

Every project has its limits, and this one is no exception. Some of these limits are the physical quality of the speaker, time limit, our ability to work together, the requirement of a passive circuit, our limited knowledge about speakers, no beforehand experience, circuitry etc. 


\subsection{Physical quality of the speaker}
This one is quite easy to explain. We have a limited amount of money so the components (capacitors, inductors, resistors etc.) will not be of top-notch quality, which means the physical quality of the speaker we are building will have some limitations. Nevertheless, we can maximize the quality of our speaker system by implementing our knowledge about electrical engineering as best as possible.

\subsection{Time limit}
Time is also a factor when we are talking about the limits of our system. The more time we have the better we can tweak, troubleshoot and solve problems. This means that we have to manage our time very well in order to deliver the maximum quality speaker, keeping in mind that we have limited time.

\subsection{Teamwork}
In order for us to complete this project we have to work as a team. Though, this is also a factor when it comes to our limitations. The quality of our project is somewhat limited by our ability to work together. This implicitly means that we have to set boundaries and communicate with each other as well as possible.

\subsection{Passive circuit}
For this project, the usage of a passive circuit for sound filters is mandatory. Also, each group is limited to the usage of 2 capacitors and 2 inductors. This results in less freedom for the group to choose their values for the capacitance and inductance, which subsequently limits the cutoff frequencies the group wants to choose for their filters. Therefore the group may not be able to make the optimal sound system, which is caused by a lack of sufficient capacitors and inductors.

\subsection{Limited knowledge}
The knowledge of the group is also limited. This will lead to the limited quality of the speaker system. There is still a lot to learn from the group members and this project is one way to do that. The members will also learn new things during the project to be able to complete the project, but not every member of the group may learn it on time, resulting in a delay caused by a group member(s).

\subsection{Experience}
Not every one of the group has experience with working in a large group on a project and for some of us, it is also the first time that we were assigned to do a project where they have to design and make a product. So there might be a difference in experience among the members of the group, which can possibly cause conflicts and delays.\\

Surely, there are more project limits, but these are the most important. It is therefore crucial to minimize the negative effects of these limits on our project whenever we can. 