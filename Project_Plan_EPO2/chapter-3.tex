\chapter{Project activities} 

In order to achieve the project's results, the group must divide the tasks well and communicate with each other clearly. In other words, work distribution has to be fair and communication should be topnotch. A detailed description of the workload distribution with a timeline will be found in the next chapter.\\
\\
Regarding communication and file management, the group will use a shared Google Drive and\\mostly WhatsApp.
In this chapter, we will briefly talk about what each group member will be doing and the roles of the meetings of the team. Furthermore, we will go over some of our limits at the second part of this chapter.

\section{Task distribution}
As mentioned, efficient task distribution is key in order for the project to end well. Below in table \ref{tab:Task distribution} you will find the subgroups of our project.

\begin{table}[h]
\centering
\setlength{\arrayrulewidth}{0.5mm}
\setlength{\tabcolsep}{19pt}
\renewcommand{\arraystretch}{1.5}
    \begin{tabular}{|c|c|}
    \hline
    Sensors     &  Darren van de Pol, Wiktor Tomanek\\
    \hline
    VHDL     &     Chantal Chen, Eojin Lim, Wilson Rong\\
    \hline
    \makecell[c]{C code and Serial Communication} & \makecell[c]{Thijs van Esch, Pieter Olyslaegers,\\ Kevin Pang, Maximiliaan Sobry}\\
    \hline
    \end{tabular}
    \captionsetup{justification=centering}
    \caption{Subgroups}
    \label{tab:Task distribution}
\end{table}


\section{Project limits}

Every project has its limits, and this one is no exception. Some of these limits are the physical quality of the robot, time limit, our ability to work together, our limited knowledge about robots, no beforehand advanced experience with robots, circuitry etc. Below we will elaborate on some of the major limits of our project.


\subsection{Physical quality of the robot}
This one is quite easy to explain. We have a limited amount of money so the components (capacitors, inductors, resistors, batteries, PCB's etc.) will not be of superb quality, which means the physical quality of the robot we are building will have some limitations. Nevertheless, we can maximize the quality of our robot system by implementing our knowledge about electrical engineering as best as possible.

\subsection{Time limit}
Time is also a factor when we are talking about the limits of our system. The more time we have the better we can tweak, troubleshoot and solve problems. This means that we have to manage our time very well in order to deliver the maximum quality robot, keeping in mind that we have limited time.


\subsection{Teamwork abilities}
In order for us to complete this project we have to work as a team. Though, this is also a factor when it comes to our limitations. The quality of our project is somewhat limited by our ability to work together. This implicitly means that we have to set boundaries and communicate with each other as well as possible. Furthermore this also means that every individual group member must do his/her part of the project accordingly.

\subsection{Limited knowledge}
The knowledge of the group is also limited. This will lead to the limited quality of the robot. There is still a lot to learn from the group members and this project is one way to do that. The members will also learn new things during the project to be able to complete the project, but not every member of the group may learn it on time, resulting in a delay caused by a group member(s).

\subsection{Communication skills}
Not everyone in the group has the same communication skills, some may be better in taking the lead and conveying progress than other for example. This can possibly lead to conflicts and therefore delays in our project. If this occurs, every group member tries his/her best to resolve the conflict, perhaps with the skills taught in ITAV.\\

Surely, there are more project limits, but these are the most important. It is therefore crucial to minimize the negative effects of these limits on our project whenever we can.